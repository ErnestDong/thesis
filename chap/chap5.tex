%!TEX root = ../thesis.tex
\chapter{稳健性检验}
表 \ref{tab:Logitresult} 中作为财务指标的 \(Z\) 值显著。wind 数据库会直接提供 \(Z\) 值给投资者使用,其 \(Z\) 值的计算公式为
\begin{equation}
	\label{eq:1}
	Z=1.2X_1+1.4X_2+3.3X_3+0.6X_4+0.999X_5
\end{equation}
其中 \(X_1\)代表营运资本/总资产,\(X_2\)代表留存收益/总资产,\(X_3\)代表息税前利润/总资产:以上变量及 \(X_5\) 均可以直接获得一致的数据;\(X_4\)代表总市值/负债总计,而当企业非上市,总市值数据不可得时,则使用股东权益代替。

式 \ref{eq:1} 中的 \(X_{5}\) 包括了营业收入/总资产,和主营业务收入回归元存在一定的相关性。可能模型中主营业务收入非常显著,因而导致 \(Z\) 值显著,而非 \(Z\) 值本身带有一定的经济学逻辑。为检验此种可能性,本文将计算 \(Z\) 值的元素展开。如果该可能性成立,则有假设:

\begin{hyp}
	\label{hyp:1}
	展开 \(Z\) 值的回归结果 \(X_i\forall i\in [1,2,3,4] \) 应不显著,而仅仅是 \(X_5\) 显著。
\end{hyp}

%!TEX root = ../thesis.tex
\begin{table}
	\captionof{table}{稳健性检验结果\label{tab:robust}}
	\begin{tabular}{p{0.1\linewidth}p{0.18\linewidth}p{0.1\linewidth}p{0.1\linewidth}p{0.1\linewidth}p{0.12\linewidth}p{0.1\linewidth}}
		\toprule
		            & \textbf{coef} & \textbf{std err} & \textbf{z} & \textbf{P$> |$z$|$} & \textbf{[0.025} & \textbf{0.975]} \\
		\midrule
		\textbf{X1} & -0.0083***    & 0.003            & -3.009     & 0.003               & -0.014          & -0.003          \\
		\textbf{X2} & -0.0037**     & 0.002            & -2.375     & 0.018               & -0.007          & -0.001          \\
		\textbf{X3} & -0.0062*      & 0.004            & -1.638     & 0.091               & -0.014          & 0.001           \\
		\textbf{X4} & -0.0064***    & 0.002            & -3.917     & 0.000               & -0.010          & -0.003          \\
		\textbf{X5} & 0.0005        & 0.001            & 0.530      & 0.596               & -0.001          & 0.002           \\
		\bottomrule
	\end{tabular}
    \qquad \small{注:括号中为异方差稳健标准误下的 Z 值;***,**,*分别表示回归系数在1\%、5\%和10\%的水平上显著,下同。}
\end{table}


稳健性经验的部分结果如表
\ref{tab:robust}
所示,与表 \ref{tab:Logitresult} 相比基本一致,且\(X_1\) 至 \(X_4\) 均显著,说明 \(Z\) 值中的其他成分均具有一定的预测能力,而非只是受到主营收入的影响导致显著,假设 H\ref{hyp:1}不成立。

事实上,\(Z\) 值反应了公司的变现能力、获利能力和财务结构,因而 Altman 使用其来对企业的运行状况进行分析。
公司的变现能力高意味着短期可以依靠出售资产回笼流动资金,避免流动性危机。典型的反例是土储集中在城郊地区的提前于行业倒下的泰禾地产,暴雷之后低价甩卖资产进一步恶化财务状况引发更多违约。
避免掏空式分红,利用留存收益和持续的利润增厚资本,可以使公司获得较强的抵御风险能力。譬如丹东港,高额分红导致资金短缺,资金缺口依靠债务回补,在融资审核趋严、实控人套现离场后留给当地违约的烂摊子。
公司的债券可以看作是卖出看跌期权,公司股权价值占比相对高,意味着公司价值距离行权价越远,KMV 等模型的违约距离也就越远,公司违约可能性相对较低。当然利用股权和债权融资都是适当的,公司不可过度偏废。综上 Z 值是一个很好的预测违约的指标。
