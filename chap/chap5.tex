%!TEX root = ../thesis.tex
\chapter{稳健性检验}
表 \ref{tab:Logitresult} 中作为财务指标的 \(Z\) 值显著。wind 数据库会直接提供 \(Z\) 值给投资者使用,其 \(Z\) 值的计算公式为
\begin{equation}
	\label{eq:1}
	Z=1.2X_1+1.4X_2+3.3X_3+0.6X_4+0.999X_5
\end{equation}
其中 \(X_1\)代表营运资本/总资产,\(X_2\)代表留存收益/总资产,\(X_3\)代表息税前利润/总资产:以上变量及 \(X_5\) 均可以直接获得一致的数据;\(X_4\)代表总市值/负债总计,而当企业非上市,总市值数据不可得时,则使用股东权益代替。

式 \ref{eq:1} 中的 \(X_{5}\) 包括了营业收入/总资产,和主营业务收入回归元存在一定的相关性。可能模型中主营业务收入非常显著,因而导致 \(Z\) 值显著,而非 \(Z\) 值本身带有一定的经济学逻辑。为检验此种可能性,本文将计算 \(Z\) 值的元素展开。如果该可能性成立,则有假设:

\begin{hyp}
	\label{hyp:1}
	展开 \(Z\) 值的回归结果 \(X_i\forall i\in [1,2,3,4] \) 应不显著,而仅仅是 \(X_5\) 显著。
\end{hyp}

\begin{center}
	\captionof{table}{稳健性检验结果\label{tab:robust}}
	\begin{tabular}{p{0.25\linewidth}p{0.2\linewidth}p{0.25\linewidth}p{0.2\linewidth}}
		\toprule
		\textbf{Dep. Variable:}   & default          & \textbf{  No. Observations:  } & 6447       \\
		\textbf{Model:}           & Probit           & \textbf{  Df Residuals:      } & 6414       \\
		\textbf{Method:}          & MLE              & \textbf{  Df Model:          } & 32         \\
		\textbf{Date:}            & Wed, 09 Mar 2022 & \textbf{  Pseudo R-squ.:     } & 0.7267     \\
		\textbf{Time:}            & 15:02:31         & \textbf{  Log-Likelihood:    } & -219.70    \\
		\textbf{converged:}       & True             & \textbf{  LL-Null:           } & -803.76    \\
		\textbf{Covariance Type:} & nonrobust        & \textbf{  LLR p-value:       } & 5.491e-225 \\
		\bottomrule
	\end{tabular}
	\begin{longtable}{p{0.18\linewidth}p{0.1\linewidth}p{0.1\linewidth}p{0.1\linewidth}p{0.1\linewidth}p{0.12\linewidth}p{0.1\linewidth}}
		\midrule
		                 & \textbf{coef} & \textbf{std err} & \textbf{z} & \textbf{P$> |$z$|$} & \textbf{[0.025} & \textbf{0.975]} \\
		\textbf{const}          & -3.8725       & 1.252            & -3.092     & 0.002               & -6.327          & -1.418          \\
		\textbf{评级\_A以上}    & -0.0382       & 0.318            & -0.120     & 0.904               & -0.661          & 0.585           \\
		\textbf{评级\_B}        & 1.4849        & 0.187            & 7.938      & 0.000               & 1.118           & 1.852           \\
		\textbf{评级\_C}        & 3.6740        & 0.228            & 16.102     & 0.000               & 3.227           & 4.121           \\
		\textbf{国有企业}       & -1.6422       & 0.398            & -4.128     & 0.000               & -2.422          & -0.862          \\
		\textbf{外资企业}       & -1.1744       & 0.467            & -2.514     & 0.012               & -2.090          & -0.259          \\
		\textbf{民营企业}       & -0.7078       & 0.385            & -1.839     & 0.066               & -1.462          & 0.046           \\
		\textbf{集体企业}       & -1.2313       & 0.534            & -2.307     & 0.021               & -2.277          & -0.185          \\
		\textbf{上市企业}       & -0.1945       & 0.192            & -1.014     & 0.310               & -0.570          & 0.181           \\
		\textbf{持有基金占比}   & -0.0783       & 0.080            & -0.973     & 0.331               & -0.236          & 0.079           \\
		\textbf{大股东持股比例} & -0.0013       & 0.003            & -0.508     & 0.611               & -0.006          & 0.004           \\
		\textbf{应付账款(万元)} & 1.409e-12     & 2.48e-12         & 0.567      & 0.570               & -3.46e-12       & 6.28e-12        \\
		\textbf{标准券折算率}   & -0.3647       & 0.651            & -0.560     & 0.575               & -1.640          & 0.911           \\
		\textbf{净资产(万元)}   & 9.314e-09     & 1.1e-08          & 0.847      & 0.397               & -1.22e-08       & 3.09e-08        \\
		\textbf{现金短债比}     & -0.1082       & 0.133            & -0.816     & 0.415               & -0.368          & 0.152           \\
		\textbf{流动性}         & -5.276e-11    & 6.63e-11         & -0.796     & 0.426               & -1.83e-10       & 7.72e-11        \\
		\textbf{政府支出/GDP}   & 1.2248        & 2.087            & 0.587      & 0.557               & -2.866          & 5.315           \\
		\textbf{SHIBOR}         & 0.3573        & 0.190            & 1.881      & 0.060               & -0.015          & 0.730           \\
		\textbf{波动率}         & 0.0341        & 0.016            & 2.135      & 0.033               & 0.003           & 0.065           \\
		\textbf{房地产政策}     & 0.7415        & 0.411            & 1.805      & 0.071               & -0.064          & 1.547           \\
		\textbf{X1}             & -0.0083       & 0.003            & -3.009     & 0.003               & -0.014          & -0.003          \\
		\textbf{X2}             & -0.0037       & 0.002            & -2.375     & 0.018               & -0.007          & -0.001          \\
		\textbf{X3}             & -0.0062       & 0.004            & -1.638     & 0.101               & -0.014          & 0.001           \\
		\textbf{X4}             & -0.0064       & 0.002            & -3.917     & 0.000               & -0.010          & -0.003          \\
		\textbf{X5}             & 0.0005        & 0.001            & 0.530      & 0.596               & -0.001          & 0.002           \\
		\bottomrule
	\end{longtable}
\end{center}


稳健性经验的部分结果如表
\ref{tab:robust}
所示,与表 \ref{tab:Logitresult} 相比基本一致,且\(X_1\) 至 \(X_4\) 均显著,说明 \(Z\) 值中的其他成分均具有一定的预测能力,而非只是受到主营收入的影响导致显著,假设 H\ref{hyp:1}不成立。

事实上,\(Z\) 值反应了公司的变现能力、获利能力和财务结构,因而 Altman 使用其来对企业的运行状况进行分析。
公司的变现能力高意味着短期可以依靠出售资产回笼流动资金,避免流动性危机。典型的反例是土储集中在城郊地区的提前于行业倒下的泰禾地产,暴雷之后低价甩卖资产进一步恶化财务状况引发更多违约。
避免掏空式分红,利用留存收益和持续的利润增厚资本,可以使公司获得较强的抵御风险能力。譬如丹东港,高额分红导致资金短缺,资金缺口依靠债务回补,在融资审核趋严、实控人套现离场后留给当地违约的烂摊子。
公司的债券可以看作是卖出看跌期权,公司股权价值占比相对高,意味着公司价值距离行权价越远,KMV 等模型的违约距离也就越远,公司违约可能性相对较低。当然利用股权和债权融资都是适当的,公司不可过度偏废。综上 Z 值是一个很好的预测违约的指标。
