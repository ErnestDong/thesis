%!TEX root = ../thesis.tex

\chapter{结论和展望}

2021 年以来,中资高收益债走入了至暗时刻,甚至对中国宏观经济造成了一定的影响。
众多过去曾经名誉较好的发行人走向了违约,有些企业官宣躺平逃废债,一些边缘的企业也被艰难的融资环境挤向了违约,以至于展期而不是甩给财务顾问重组都成了对投资人还算友好的债务处理方式:但更多的企业仍在苦苦支撑不躺平,努力产生现金流维持运转。未来的城投债压力和经济增速逐步放缓的背景下,城投债如果违约亦将是打破预期的大冲击事件,也会对信用债市场造成冲击,影响非常多的实体企业,甚至于使本不该倒下的企业倒掉。

诚然违约是且应当是经济中的正常现象,然则前事不忘,后事之师。我们可以看到,就目前而言,企业保持以 Z 值为代表的健康的财务状况,扩大主营业务,维持一定的营运资本和息税前利润率,减少高分红以增厚企业收益,规避高杠杆,提升资产周转率和利用率,是可以显著降低违约概率的措施。政府应当保持稳健的政策,提供稳定的政策和融资环境,引导产业发展时不应过度扶持。在这个大变革的时代,虽前路艰险道阻且长,我们仍能对金融市场的稳定保持信心。
