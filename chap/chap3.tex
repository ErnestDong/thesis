%!TEX root = ../thesis.tex

\chapter{研究设计}
\section{样本筛选与数据来源}
由于 2014 年以前有“刚兑神话”,企业无力负担债务时承销商、担保方或是股东会选择垫付以避免公开市场违约,因此我们选择的样本覆盖了 2014 年初至今发行的信用债发行人。考虑到我国金融机构和非金融机构之间存在很大的监管差异。金融机构特别是银行、保险等行业在出现风险时,往往由于涉及面众多,牵涉民生较广,可能会引发系统性风险,政府往往会及时采取手段控制风险,如接管安邦保险、包商银行、天安财险/人寿等金融机构。在真实的案例中,亦只有天安人寿一家有展期债券,其他保险公司和银行券商均没有违约。因此在样本筛选过程中,我们排除掉商业银行、保险公司、证券公司类的金融机构。

城投公司是我国一个独特的存在。截止 2022 年 2 月,我国信用债券市场中企业债、公司债、中期票据、短融、超短融以及定向工具存量 66.5 万亿,而城投债存量 13 万亿。
大多数区县级城投公司以及少数市级城投公司财务状况都很不健康:负债率居高不下,财政回款慢,公益性质大盈利能力不强。但迄今为止真正意义上的城投公司违约尚未出现。\Textcite{钟辉勇2016城投债的担保可信吗}就指出城投债可能存在隐性政府背书兜底。例如 2020 年底受永煤信用事件的冲击,投资者对违约的担忧上升、风险偏好下降,出现了“抱团”城投的现象。
2021 年 7 月银保监发 [2021]15 号文指出,各银行保险机构要严格执行地方政府融资相关政策要求,打消财政兜底幻觉,强化合规管理、尽职调查,不得以任何形式新增地方政府隐性债务。或许在将来城投公司违约将逐步正常化,但目前而言城投公司违约的影响因素尚不清晰且与其他企业区别较大,因此我们在数据中亦排除城投债。

不同评级公司的评级标准不同,划分的等级也有所不一。但大多评级都是分为 A、B、C 三档,每档中划分三个主要层次,最后辅以一个正负表示是否略高或略低于标准。在回归中我们将评级为最高 AAA 的划为一档,AA 及 A 划分为一档,BBB 以下至 B 以及 CCC 以下至 C 为一档。这么做的原因如 \ref{sec:zs} 中所述,我国评价公司评级比较扭曲,集中在 AAA 和 AA 。不同档次间违约概率差别不一定比较大,大量债券被划分为了高等级债券。为了降低评级不准确的影响,本文将评级按照上述标准分类。

清洗后的样本中最终包含 6412 家发行人。本文所用数据来自 wind 数据库。

\section{变量选择与模型设计}

\subsection{被解释变量}
本文所使用被解释变量违约的概念为:主体限定在在大陆地区发行信用债的企业,形式包括展期、未按时兑付本金或回售款或利息、提前到期未兑付、触发交叉违约、技术性违约、担保违约和场外兑付等一切与债券发行时合同相抵触的影响债券现金流的行为。被解释变量 \(default\) 为发行人是否违约,违约为 1 ,不违约为 0。

\subsection{解释变量}
本文选取的解释变量包括
\begin{enumerate}
	\item 微观层面
	      \begin{itemize}
		      \item 公司:企业性质、是否上市、持有基金占比、大股东持股比例
		      \item 经营:主营业务收入、应付账款、标准券
		      \item 财务:净资产、现金短债比、Z值
		      \item 评级
	      \end{itemize}
	\item 中观层面:流动性、房地产政策
	\item 宏观层面:政府支出/GDP、SHIBOR 利率、波动率
\end{enumerate}

\subsection{控制变量}
我们在回归中加入年份作为控制变量,以控制模型未关注到的事件冲击。

\begin{table}[ht]
	\centering
	\caption{主要变量及其定义一览}
	\begin{tabular}{cccc}
		\hline
		\textbf{变量类型}  & \textbf{变量名称} & \textbf{变量符号}   & \textbf{定义}       \\ \hline
		\textbf{被解释变量} & 违约            & \(Default\)     & 违约为1,否则为0         \\\hline
		\textbf{解释变量}  & 企业性质          & \(Enterprise\)  & 国有、集体、外资、民营       \\
		               & 是否上市          & \(Listed\)      & 主体是否在境内外上市        \\
		               & 持有基金占比        & \(Fund\)        & 债券持有人中公募基金占比      \\
		               & 大股东持股比例       & \(Shareholder\) & 第一大股东及一致行动人持股比例   \\
		               & 主营业务收入        & \(Income\)      & 主营业务收入            \\
		               & 应付账款          & \(Payable\)     & 应付账款(万元)          \\
		               & 标准券           & \(Conversion\)  & 标准券折算率            \\
		               & 净资产(万元)       & \(Assets\)      & 净资产(万元)           \\
		               & 现金短债比         & \(Cash\)        & 货币资金/短期及到期有息债务    \\
		               & Z 值           & \(Z\)           & Altman's Z        \\
		               & 最新评级          & \(Rating\)      & 按照 A 以上、A、B、C分档   \\
		               & 流动性           & \(Liquidity\)   & 债券成交量             \\
		               & 房地产政策         & \(Estate\)      & 发债日是否在“三条红线”政策实施后 \\
		               & 政府支出/GDP      & \(Fiscal\)      & 政府支出占GDP比重        \\
		               & SHIBOR        & \(Monetary\)    & SHIBOR 一年期利率      \\
		               & 波动率           & \(Volatility\)  & 上证50期权波动率         \\\hline
		\textbf{控制变量}  & 发债年份          & \(Year\)        & 最新发债日期            \\
	\end{tabular}
	\label{tab:symbols}
\end{table}


\subsection{基本模型}
本文使用 Logit 回归模型对系数进行估计,并根据回归结果来分析不同因素对公司违约概率的影响:
\begin{equation}
  Default = \beta_0+ \beta_1Year+\sum_{i\in Micro}\beta_i\cdot i+\sum_{j\in Middle}\beta_j\cdot j+\sum_{k\in Macro}\beta_k\cdot k
\end{equation}
其中
\begin{eqnarray}
  Micro&=&\{Enterprise, Listed, Fund, Shareholder, Income, payable,\nonumber\\
          & & Conversion, Assets, Cash, Z, Rating\} \nonumber\\
	Middle&=&\{Liquidity, Estate\} \nonumber\\
	Macro&=&\{Fiscal, Monetary, Volatility\} \nonumber
\end{eqnarray}
\section{变量描述性统计}
%!TEX root = ../thesis.tex
全局性描述结果如表\ref{globaldesc}与表 \ref{virtual}\footnote{所有变量均为虚拟变量} 所示

\begin{table}[ht]
	\caption{主要变量及其描述性统计 I }
	\label{globaldesc}
	\begin{tabular}{p{0.15\linewidth}p{0.1\linewidth}p{0.15\linewidth}p{0.15\linewidth}p{0.15\linewidth}p{0.15\linewidth}}
		\toprule
		\textbf{变量名}    & \textbf{样本数} & \textbf{平均值} & \textbf{标准差} & \textbf{最小值} & \textbf{最大值} \\ \midrule
		\(Fiscal\)      & 6412         & 0.211069     & 0.04211      & 0.092627     & 0.286154     \\
		\(Monetary\)    & 6412         & 3.117273     & 0.64046      & 1.885274     & 4.774387     \\
		\(Volatility\)  & 6412         & 23.51951     & 4.769809     & 11.29499     & 41.50071     \\
		\(Liquidity\)   & 6412         & 8.58E+08     & 6E+09        & 0            & 1.61E+11     \\
		\(Fund\)        & 6412         & 0.809421     & 4.931441     & 0            & 119.9634     \\
		\(Shareholder\) & 6412         & 71.21923     & 33.64897     & 0            & 100          \\
		\(Income\)      & 6412         & 1987012      & 8875087      & 1.622648     & 2.67E+08     \\
		\(Payable\)     & 6412         & 3.94E+09     & 1.93E+10     & 0            & 5.02E+11     \\
		\(Conversion\)  & 6412         & 0.021204     & 0.123949     & 0            & 0.99         \\
		\(Assets\)      & 6412         & 1877755      & 9023818      & -3594115     & 5.64E+08     \\
		\(Cash\)        & 6412         & 0.99765      & 37.29003     & 0            & 2982.515     \\
		\(Z\)           & 6412         & 1.373761     & 3.01916      & -56.178      & 202.0643     \\
		\bottomrule
	\end{tabular}
\end{table}
\begin{table}[ht]
	\caption{主要变量及其描述性统计 II }
	\label{virtual}
	\begin{tabular}{p{0.3\linewidth}p{0.3\linewidth}p{0.3\linewidth}}
		\toprule
		\textbf{变量名}              & \textbf{样本数} & \textbf{占比} \\ \midrule
		\(R_0=Rating is AA\)      & 1127         & 0.175764    \\
		\(R_1=Rating is A\)       & 5082         & 0.7925764   \\
		\(R_2=Rating is B\)       & 78           & 0.012165    \\
		\(R_3=Rating is C\)       & 125          & 0.019495    \\
		\(E_0=Enterprise is\)民营企业 & 1146         & 0.178727    \\
		\(E_1=Enterprise is\)国有企业 & 4938         & 0.770119    \\
		\(E_2=Enterprise is\)外资企业 & 178          & 0.02776     \\
		\(E_3=Enterprise is\)集体企业 & 126          & 0.019651    \\
		\(Listed=1\)              & 1393         & 0.217249    \\
		\(Estate=1\)              & 173          & 0.026981    \\
		\(Y_1=Year is 2015\)      & 320          & 0.049906    \\
		\(Y_2=Year is 2016\)      & 418          & 0.06519     \\
		\(Y_3=Year is 2017\)      & 346          & 0.053961    \\
		\(Y_4=Year is 2018\)      & 307          & 0.047879    \\
		\(Y_5=Year is 2019\)      & 451          & 0.070337    \\
		\(Y_6=Year is 2020\)      & 927          & 0.144573    \\
		\(Y_7=Year is 2021\)      & 2767         & 0.431535    \\
		\(Y_8=Year is 2022\)      & 674          & 0.105115    \\
		\(default=1\)             & 173          & 0.026981    \\
		\bottomrule
	\end{tabular}\\
\end{table}

