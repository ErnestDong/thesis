%!TEX root = ../thesis.tex
\begin{cabstract}
	偶发的债券违约是经济体排除不稳定因素、自发降低系统性风险的正常现象,维持“刚兑”的债券市场反而是不健康的。但违约结果造成的影响是多方面的,处理得当可以使短期遭受困难的企业焕发新生,处理不当则可能影响金融市场稳定,进而对实体经济造成巨大影响。
	本文从债券违约的影响因素出发,以2014年来债券发行数据为样本,对我国债券市场上违约发行人的共性特征进行了分析,抽离出导致企业违约的重要因素,并通过机器学习的方式佐证和拓展了计量模型,以期对债券市场违约有一个比较好的刻画。研究发现:企业的经营收入、财务效率与宏观经济的波动率对企业债券违约有显著影响。
\end{cabstract}
\begin{eabstract}
	Occasional bond default is a normal phenomenon for an economy to eliminate unstable factors and spontaneously reduce systemic risks. It is unhealthy to maintain a bail out bond market. However, the impact of the default results is multi-faceted. Proper handling can rejuvenate companies that have suffered short-term difficulties. Improper handling may affect the stability of the financial market, and thus have a huge impact on the real economy.
	Starting from the influencing factors of bond defaults and taking the bond issuance data since 2014 as a sample, this paper analyzes the common characteristics of default issuers in Chinese bond market, extracts important factors that lead to corporate defaults, and uses machine learning. And extended the econometric model in order to have a better characterization of the bond market default. The study found that: the operating income, financial efficiency and macroeconomic volatility of enterprises have a significant impact on corporate bond defaults.
\end{eabstract}
