%!TEX root = ../thesis.tex
\begin{cabstract}
	偶发的债券违约是经济体排除不稳定因素、自发降低系统性风险的正常现象,维持“刚兑”的债券市场反而是不健康的。但违约结果造成的影响是多方面的,处理得当可以使短期遭受困难的企业焕发新生,处理不当则可能影响金融市场稳定,进而对实体经济造成巨大影响。
	本文计划从债券违约影响因素出发,使用计量手段描述我国债券市场上违约发行人的共性特征,抽离出重要的导致企业违约的关键因素,并通过机器学习的方式佐证和拓展计量模型,以期对债券市场违约有一个比较好的刻画。
\end{cabstract}
\begin{eabstract}
	Occasional bond defaults are a normal phenomenon for an economy to eliminate unstable factors and spontaneously reduce systemic risks, and it is unusual to maintain a bail out bond market. However, the impact of the default results is multi-faceted. A proper handling can rejuvenate companies that have suffered short-term difficulties. But an improper handling may affect the stability of the financial market and have a huge impact on the economy.
	This paper plans to start from the influencing factors of bond defaults. We use measurement methods to describe the common characteristics of default issuers in China bond market, extract important key factors that lead to corporate defaults. And finally we use machine learning to corroborate and expand our model in order to  have a better characterization of bond market defaults.
\end{eabstract}
