\begin{table}[h]
	\centering
	\begin{tabular}{|p{0.2\linewidth}p{0.2\linewidth}p{0.2\linewidth}p{0.2\linewidth}|}
		\hline
		\multicolumn{4}{|c|}{\large\textbf{本科学生毕业论文指导教师成绩评定表}}                                                      \\
		\multicolumn{4}{|c|}{}                                                                                        \\
		\hline
		\multicolumn{1}{|l|}{}         & \multicolumn{1}{|l|}{} & \multicolumn{1}{|l|}{}              & \multicolumn{1}{|l|}{} \\
		\multicolumn{1}{|l|}{学生姓名} & \multicolumn{1}{|l|}{董晨阳} & \multicolumn{1}{|l|}{学号}          & \multicolumn{1}{|l|}{1800015446} \\
		\multicolumn{1}{|l|}{}         & \multicolumn{1}{|l|}{} & \multicolumn{1}{|l|}{}              & \multicolumn{1}{|l|}{} \\
		\hline
		\multicolumn{1}{|l|}{}         &                        &                                     &                        \\
		\multicolumn{1}{|l|}{论文题目} &  \multicolumn{3}{|l|}{我国债券市场违约特征分析}                        \\
		\multicolumn{1}{|l|}{}         &                        &                                     &                        \\
		\hline
		\multicolumn{4}{|l|}{}                                                                                                 \\
		\multicolumn{4}{|l|}{指导教师评阅意见:}                                                                               \\
		\multicolumn{4}{|l|}{}                                                                                                 \\
		\multicolumn{4}{|l|}{}                                                                                                 \\
      \multicolumn{4}{|l|}{
      \qquad 债券违约受到多方面因素的影响,如何从全面视角出发对债券违约特征

      }\\
      \multicolumn{4}{|l|}{
      进行综合评价,对于债券违约的预警等具有重要意义。论文《我国债券市场违

      }\\
      \multicolumn{4}{|l|}{
      约特征分析》以此为出发点,基于大量相关数据,进行了实证研究。本文研究

      }\\
      \multicolumn{4}{|l|}{
      方法得当,结构清晰、重点突出,逻辑严密,引用资料准确、丰富,结论正确,

      }\\
      \multicolumn{4}{|l|}{
      文中用语和图表格式等符合本科论文的要求。作者在前人研究的基础上,从宏

      }\\
      \multicolumn{4}{|l|}{
      观、中观和微观三个层面出发选取指标、构建模型,对债券违约进行综合评价,

      }\\
      \multicolumn{4}{|l|}{
      并通过计量及机器学习方法进行了定量分析,具有一定的创新性,反映出该生

      }\\
      \multicolumn{4}{|l|}{
      对此问题经过了深入的思考,具有一定的研究能力,论文达到本科论文的要求。}\\
		\multicolumn{4}{|l|}{}                                                                                                 \\
		\multicolumn{4}{|l|}{}                                                                                                 \\
		\multicolumn{4}{|l|}{}                                                                                                 \\
		\multicolumn{4}{|l|}{}                                                                                                 \\
		\multicolumn{4}{|l|}{}                                                                                                 \\
		\multicolumn{4}{|l|}{}                                                                                                 \\
		\multicolumn{4}{|l|}{}                                                                                                 \\
		\multicolumn{4}{|l|}{}                                                                                                 \\
		\multicolumn{4}{|l|}{}                                                                                                 \\
		\multicolumn{4}{|l|}{}                                                                                                 \\
		\multicolumn{4}{|l|}{}                                                                                                 \\
		\multicolumn{4}{|l|}{}                                                                                                 \\
		\multicolumn{4}{|l|}{}                                                                                                 \\
		\multicolumn{4}{|l|}{}                                                                                                 \\
		\hline
		\multicolumn{4}{|l|}{}                                                                                                 \\
		\multicolumn{4}{|l|}{指导教师建议论文成绩为:}                                                                         \\
		\multicolumn{4}{|l|}{}                                                                                                 \\
		                               &                        & \multicolumn{2}{l|}{指导教师签名:}                          \\
		                               &                        & \multicolumn{2}{l|}{日期:}                                    \\
		\multicolumn{4}{|l|}{}                                                                                                 \\
		\hline
	\end{tabular}
\end{table}
